\documentclass[12pt,a4paper]{article}
\usepackage[utf8]{inputenc}
\usepackage[margin=2.5cm]{geometry}
\usepackage{amsmath,amsfonts,amssymb}
\usepackage{graphicx}
\usepackage{float}
\usepackage{caption}
\usepackage{subcaption}
\usepackage{hyperref}
\usepackage{bookmark}

\begin{document}

% ---- Cover Page ----
\begin{titlepage}
    \centering
    \vspace*{1.5cm}

    {\Large \textbf{Data-Driven Methods for Engineers}}\\[0.4cm]
    {\Large \textbf{(MECH0107)}}\\[0.4cm]
    {\Large \textbf{- Coursework 1 -}}\\[1.5cm]

    {\large 2025 -- 2026}

    \vspace{2cm}

    \raggedright
    \tableofcontents

\end{titlepage}
\newpage

% =========================================================================
% 1. Introduction and Problem Statement
% =========================================================================
\section{Introduction and Problem Statement}
% Brief description of the mass-spring-pendulum system and its physics
% What the datasets contain (3 cameras, different angles)
% Objectives: extract motion -> analyse frequencies -> identify modes -> interpret physics
% Short workflow overview

% =========================================================================
% 2. Methodology
% =========================================================================
\section{Methodology}

\subsection{Motion Tracking (Image Processing)}
% Pipeline: grayscale conversion -> Gaussian filtering -> thresholding -> morphological cleanup -> centroid extraction
% Why brightness-based for Cam 1 & 3, yellow channel for Cam 2
% Camera 3 rotation correction
% Justify technique: accuracy, computational cost, robustness to noise
% Reference Lecture 4 techniques

\subsection{Time--Frequency Analysis (Gabor Transform)}
% The Gabor transform formula and what it does
% Window width and the Heisenberg trade-off
% Frequency in cycles/frame (no frame rate metadata)

\subsection{Dimensionality Reduction (SVD)}
% Data matrix construction (6 rows x n columns)
% SVD decomposition: A = U Sigma V*
% Why SVD over PCA (no covariance matrix needed, numerically stable, works directly on data matrix)

% =========================================================================
% 3. Results
% =========================================================================
\section{Results}

\subsection{Tracked Displacement Data}
% Displacement plots from all 3 cameras
% Camera 1 overhead shows pendulum swing damping out, sustained vertical oscillation

\subsection{Spectrograms and Frequency Evolution}
% Global FFT: two distinct frequency peaks
% Spectrograms at different Gabor widths (the comparison figure)
% Dominant frequency evolution over time

\subsection{SVD Mode Decomposition}
% Singular value spectrum and cumulative variance
% First 3 modes and their temporal behaviour
% Spatial structure (which cameras/axes contribute to each mode)
% Low-rank reconstruction

% =========================================================================
% 4. Discussion
% =========================================================================
\section{Discussion}

\subsection{Physical Interpretation of Modes}
% Mode 1: spring oscillation (sustained, vertical-weighted)
% Mode 3: pendulum swing (damps quickly, horizontal-weighted)
% Mode 2/3: perspective effects, possible second swing direction

\subsection{Mode Coupling and Nonlinearity}
% Amplitude modulation in vertical signal suggests energy transfer
% Spectrogram shows frequency components aren't perfectly constant
% 3 modes needed instead of 2 -- evidence of nonlinear interaction

\subsection{Simplified Analytical Model}
% Small-angle equations for pendulum + spring
% Expected frequencies vs observed frequencies
% Discrepancies: damping, geometric nonlinearity, 2D projection limitations

\subsection{Sources of Error and Improvements}
% Tracking noise, camera calibration, no frame rate metadata
% Suggested improvements: calibrated stereo cameras, 3D reconstruction, higher frame rate, physical markers

% =========================================================================
% 5. Conclusion
% =========================================================================
\section{Conclusion}
% System has 2-3 effective DOF
% SVD successfully separated spring and pendulum modes
% Nonlinear coupling evidenced by energy transfer and extra modes
% Limitations of 2D tracking motivate 3D reconstruction in future work

\end{document}